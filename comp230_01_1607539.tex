% Please do not change the document class
\documentclass{scrartcl}

% Please do not change these packages
\usepackage[hidelinks]{hyperref}
\usepackage[none]{hyphenat}
\usepackage{setspace}
\doublespace

% You may add additional packages here
\usepackage{amsmath}
\usepackage{graphicx}
\usepackage{url}
\usepackage[autostyle]{csquotes}

% Please include a clear, concise, and descriptive title
\title{What are the risks and rewards associated with nurturing a modding community?}

% Please do not change the subtitle
\subtitle{COMP230}

% Please put your student number in the author field
\author{Student number: 1607539}

\begin{document}

\maketitle

Game developers can encourage modifications (mods) by releasing the game's source code to the public. An often repeated example is the case of \textit{Doom} (1993) in which the media files were separated from the main program and made accessible for users \cite{sotamaa2010game}. This is historically celebrated as a consequence of mutual respect \cite{hong2013game}. Modders use these toolsets to build mods of a number of different scales, from the massive ``total conversions'' accomplished by teams of modders who remake entire games with new artwork, storylines, and gameplay to smaller mods such as ``skin'' (aesthetic tweaks to specific graphic models) \cite{hong2014becoming}.  

When Todd Howard, the executive producer for Bethesda was asked why some companies are not making their games modifiable, he replied, ``I don't understand why they don't, I think it makes your games better'' \cite{todd2012}. As a business decision, choosing to encourage modding allows continuous free new content beyond the designs of the commercial developers \cite{postigo2007mods}. But, leaving your intellectual property open to the intent of the community can result in morally questionable content. This paper outlines the benefits of supporting a modding community, then examines each risk, before advocating the overall benefit of this development choice.

\subsection*{Modding culture benefits}

Modders are an evolution of the consumer, reconfiguring the position of the user. No longer simply purchasing the game as a product, but being actively engaged in its continued development and marketing \cite{herman2006your}. With \textit{Skyrim}, modders constitute a source of free work upon which Bethesda have capitalised \cite{hong2013game}. If the average large mod requires more than 1,000 work hours to complete, we can estimate the development cost saved by referencing salaries for the profession \cite{postigo2007mods}. While Modders unpaid labour can make their work prone to industry appropriation, it can also serve as part of a portfolio \cite{nieborg2008mod}. id Software, the company which produced \textit{DOOM}, discovered many of their current employees and development partners based on mods that were created and distributed over the Internet \cite{hong2014becoming}. 

Linden Labs (\textit{Second Life}) also embrace the idea of player-generated intellectual property. President and CEO Phillip Rosedale said ``We allow people to create a world which will be thousands of times more compelling than we could create ourselves'' \cite{herman2006your}. Although, now the game world is rife with bootlegged videos and audio streams of popular music \cite{herman2006your}. \textit{Wikipedia} protects itself by using scripts that perform automated tasks of policing content and correcting vandalism \cite{hong2014becoming}, but censorship imposes a cost that must be part of a consequentialist moral calculation \cite{schulzke2010defending}.

\subsection*{Modding culture risks}

The motivation to produce a mod can vary from purely aesthetic to a more political one. Projects focusing on particular historical events (\textit{Operation Gulf War Crisis, Battle over Hokkaido, 1982: Flashpoint in the Falklands}) often have a national emphasis. Understandably these kinds of themes tend to open up political debates among mod makers \cite{sotamaa2010game}, which may not align with the developers' vision for their game.

Next consider that the \textit{Fallout 3} mod website Nexus now hosts more than 7500 mods \cite{bostan2010explorations}, ranging from sexually revealing armours, nude female models, and the ``killable children'' mod \cite{bostan2010explorations}. Bethesda had already decided that the child characters in their game could not be harmed, so by creating this feature the modders are directly contradicting the developers' ethics. Perhaps this subverting of a given design is exactly what a modification should be. But at what cost? A \textit{GTA V} developer famously cost Take-Two {\$}24 million when the hidden `hot-coffee' minigame was discovered causing suppliers to remove the game from sale \cite{hotcoffee1}.

We should recognize that in the modding community there is resistance to governance. That there are those who choose to ignore the license terms and conditions that come with closed, restricted games \cite{scacchi2010computer}. Even a strict modding policy that outlines exactly what is and is not permissible, such as Rockstar has with \textit{GTA V}, can result in controversial cease and desist orders \cite{gta1}. Nintendo are also extremely protective of their intellectual property, regularly serving takedown notices to modders who dilute the \textit{Mario}, \textit{Zelda} \cite{nintendo1}, or \textit{Pok{\'e}mon} \cite{pokemon1} brands. Value is derived from sales of software, and a good reputation can accrue with an application's technical integrity \cite{herman2006your}. However, this can backfire against the developer. For example, copyright disputes between corporations and consumers in the music industry have tarnished the opinions of consumers towards certain bands (i.e. \textit{Metallica}) perceived to be overzealous in enforcing their intellectual property rights \cite{herman2006your}. 

Mods are considered to be ``hacker art'', an artistic endeavour, and a creative outlet \cite{sotamaa2010game}. Although, one of the oldest fears about art is that it may corrupt observers and lead them to immorality \cite{schulzke2009moral}. In games, the amount of unethical violence usually depends heavily on how a player chooses to act \cite{schulzke2010defending}. Ethically notable games either attempt to make the player feel responsible for the decisions they make in the game, encode an ethical system and require the player to follow it in order to succeed, or provide players with situations in which their understanding of the ethical system is challenged \cite{zagal2009ethically}. Then is a game that only allows a good course of action be ethically correct if it does not allow free will? Considering the opposite, is Danny Ledonne's `linear' game \textit{Super Columbine Massacre RPG!} condemned simply because it is a game about a serious and emotional topic? Perhaps the act of playing a game trivializes the issues it tackles and thus, renders any game about a serious topic inherently unethical \cite{zagal2009ethically}. Game developers should consider how the player can behave ethically, more than the social impact of the narrative. When considering the public image of a game's brand, question if it is ethical for this game to exist? Should this game have been created in the first place? What in-game actions are defined as ``good'' by the game \cite{zagal2009ethically}? A mod, either as an extension or total conversion, should be subject to the same rigor. Serious attention must be given to the gap that may exist between modders' design desires and corporate interest \cite{taylor2002whose}.

\subsection*{Conclusion}

Despite the risk that a mod may not align politically or ethically with the original design of the game, the benefits of a community led, fan made, modding culture are entirely convincing. As a business model, inventive mods can not only strengthen the unit sales of the games from which they originated; their innovative playing format can author entirely new genres of games, produce content for free, fix bugs, and add patches \cite{hong2014becoming}.  Legal remedies exist to protect copyrighted material \cite{copyright1988} \cite{trademarks1994} and takedown notices can be served, but developers must be aware that community backlash can adversely affect profits and reputation. A future study will further examine the legal recourse that developers can and have taken against innappropriate mods.

\bibliographystyle{ieeetran}
\bibliography{references}

\end{document}

over file sharing and sampling 

, and offer recruitment opportunities

For support using an existing trademark, modders must apply for a license agreement \cite{gamerlaw1}.

In 1999, two young enthusiastic amateur developers, Minh `\textit{Gooseman}' Le and Jess `\textit{Cliffe}' Cliffe, and their team brought us the mod \textit{Counter-Strike}, based on the popular first-person shooter game \textit{Half-life} (Valve, 1998) \cite{nieborg2008mod}. 

Modding communities lower production costs by producing content for free, fixing bugs, and adding patches \cite{hong2014becoming}. But potential conflicts may arise as questions of ownership in derivative works of copyrighted content, pushing some of these development relationships into courts \cite{postigo2007mods}.

One player spent 90 min playing GTA 4 and only killed two people \cite{schulzke2010defending}


